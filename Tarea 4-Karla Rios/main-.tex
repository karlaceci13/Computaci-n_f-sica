\documentclass[a4peper, 12pt]{article}
\usepackage[utf8]{inputenc}
\usepackage{amsmath}
\usepackage{amssymb}
\usepackage{mathrsfs}
\usepackage{latexsym}
\usepackage{dsfont}
\usepackage{amsfonts}
\usepackage{array}
\usepackage{fancyhdr}
\usepackage{xcolor}
\pagestyle{fancy}

\lhead{Rios Lara Karla Cecilia Valeria}
\rhead{Física y Matemáticas}
\begin{document}
\begin{center}
\textit{\textbf{\huge{Formulitas :)}}}    
\end{center}
\section{Física}
\subsection{Movimiento rectilíneo uniformemente variado}

$a= \frac {V_{f} - V_{i}}{D}$

\begin{itemize}
    \item a = aceleración $(\frac{m}{s^2})$
    \item $V_{f}$ = Velocidad final $(\frac{m}{s})$
    \item $V_{i}$ = Velocidad inicial $(\frac{m}{s})$
    \item D = Distancia (m)
\end{itemize}
\subsection{Tiro parabólico}

$H=\frac{Vo^2 \cdot Sen^2 \theta}{2g}$
\begin{itemize}

\item H: altura máxima (m)

\item v0: velocidad inicial $(\frac{m}{s})$

\item $\theta$: ángulo de la dirección del lanzamiento

\item g: aceleración de la gravedad $(\frac{m}{s^2})$

\end{itemize}

\subsection{Caída libre}

$y= V_{0} \cdot t + \frac{1}{2} g \cdot t$
\begin{itemize}
    \item  y = distancia (m)
    \item  $V_{0}$ = Velocidad inicial $\frac{m}{s}$
    \item  t =tiempo (s)
    \item  g = gravedad $(9.81 \frac{m}{s^2})$
\end{itemize}

\subsection{Trabajo, energía y potencia}

$Ec=\frac{1}{2} m\cdot v^2$
\begin{itemize}
    \item Ec: energía cinética (J)
    \item m: masa (kg)
    \item v: velocidad $(\frac{m}{s})$
\end{itemize}

$P= \frac{T}{\Delta t}$
\begin{itemize}
    \item P: potencia (w)
    \item T: trabajo (J)
    \item $\Delta$ t: intervalo de tiempo (s)
\end{itemize}

\subsection{Gravitación universal}

\textcolor{magenta}{$F_{G}= \frac{G \cdot M_{1} \cdot M_{2}}{d^2}$}
\begin{itemize}
    \item[\#] $F_{G}$ : fuerza gravitacional (N)
    \item[\#] G: constante de gravitación universal $(\frac{N.m^2}{kg^2}$
    \item[\#] $M_{1}$ : masa del cuerpo 1 (kg)
    \item[\#] $M_{2}$ : masa del cuerpo 2 (kg)
    \item[\#] d: distancia (m)
\end{itemize}

\subsection{Termodinámica}

$\Delta A= A_{0} \cdot \beta \cdot \Delta T$

\begin{itemize}
    \item[\#] $\Delta$A : dilatación superficial $(m^2)$
    \item[\#] $A_{0}$ : área inicial $(m^2)$
    \item[\#] $\beta$ : coeficiente de dilatación superficial $(C ^\circ -1)$
    \item[\#] $\Delta$ : variación de temperatura  $(C ^\circ )$ 
\end{itemize}

\subsection{Electromagnetismo}

$\epsilon= \frac{\Delta \phi}{\Delta t}$

\begin{itemize}
    \item[\#] $\epsilon$ : fem inducida (V)
    \item[\#] $\Delta \phi$ : variación del flujo magnético (Wb)
    \item[\#] $\Delta$ t: intervalo de tiempo (s)
\end{itemize}

\subsection{ Ecuación De Broglie}

$\lambda = \frac{h}{p}$ $\lambda = \frac{h}{m \cdot v}$
\begin{itemize}
    \item [\#] $\lambda$ = longitud de onda 
    \item[\#]  h= constante de plank
    \item [\#] p= movimiento de la partícula
    \item[\#]  m= masa de del corpúsculo
    \item[\#]  v= velocidad del corpúsculo
\end{itemize}

\subsection{ 2da Ley de Newton}

$F= m \cdot a$

\begin{itemize}
    \item[\#]  F = fuerza necesaria para mover un cuerpo u objeto (N)
    \item[\#]  m = masa de un cuerpo (kg)
    \item[\#]  a = unidad de aceleración  $(\frac{m}{s^2})$
\end{itemize}

\section{Matemáticas}

\subsection{Ecuación de segundo grado}

$x = \frac{-b\pm \sqrt{b^2 - 4ac}}{2a}$
\begin{itemize}
    \item[$\bigstar$] La fórmula cuadrática nos ayuda a resolver cualquier ecuación cuadrática. Primero ponemos la ecuación en la forma $ax^2 + bx + c = 0$, donde a, b y c son coeficientes.
\end{itemize}

\subsection{Binomio al cuadrado}

$(a + b )^2  = a^2 + 2ab + b^2$
\begin{itemize}
    \item[$\bigstar$] Un binomio al cuadrado (suma) es igual es igual al cuadrado del primer término, más el doble producto del primero por el segundo más el cuadrado segundo.
\end{itemize}
\subsection{Binomios conjugados}
$(a + b)(a - b)= a^2 - b^2 $
\begin{itemize}
    \item[$\bigstar$] El producto de binomios conjugados, es decir la suma de dos cantidades multiplicadas por su diferencia es igual al cuadrado de la primera cantidad menos el cuadrado de la segunda.
\end{itemize}

\subsection{Teorema de pitágoras}

$c^2 = a^2  + b^2 $
\begin{itemize}
    \item[$\bigstar$] El teorema de Pitágoras establece que en todo triángulo rectángulo, el cuadrado de la longitud de la hipotenusa es igual a la suma de los cuadrados de las respectivas longitudes de los catetos.
\end{itemize}

\subsection{Ley de los senos}
 
 $\frac{a}{sen \alpha}$ $\frac{b}{sen \beta}$ $\frac{c}{sen \gamma}$

\begin{itemize}
    \item[$\bigstar$] La ley de los senos es la relación entre los lados y ángulos de triángulos no rectángulos (oblicuos). Simplemente, establece que la relación de la longitud de un lado de un triángulo al seno del ángulo opuesto a ese lado es igual para todos los lados y ángulos en un triángulo dado.
\end{itemize}

\subsection{Cálculo Integral}

\begin{itemize}
    \item[$\clubsuit$] Sean a,k, y C constantes (números reales) y consideremos a u = u(x) como función de x y a u' = u'(x) como la derivada de u, entonces se cumplen las siguientes igualdades de integración:
\end{itemize}
\begin{itemize}
    \item[$\blacktriangle$] $\int u^n \cdot u'dx= \frac{u^n+1}{n+1}+C, n\neq
-1$
    \item[$\blacktriangle$] $\int cos(u) \cdot u'dx = sin (u) + C $
\end{itemize}

\subsection{Definición de derivada}

\textcolor{blue}{$$\frac{df(x)}{dx}= \lim_{h \rightarrow 0} \frac{f(x-h)-f(x)}{h}$$}
\begin{itemize}
    \item[$\clubsuit$] $m_{tan}$: es la pendiente de la tangente a f(x) en un punto
\end{itemize}

\subsection{Notación sigma (Sumatoria)}
\begin{itemize}
    \item[$\square$] Al calcular las áreas de regiones con frecuencia necesitamos considerar las suma de los primeros n enteros positivos,así como las sumas de sus cuadrados, cubos, etc. 
\end{itemize}
\begin{itemize}
    \item[$\diamondsuit$] $$\sum_{i=0}^{n} i = \frac{n (n+1)}{2}$$
    \item[$\diamondsuit$] $$\sum_{i=0}^{n} i^2 = \frac{n(n+1)(2n+1}{6}$$
\end{itemize}

\subsection{Áreas y Volúmenes}

$A= \pi \cdot R (R+g)$

\begin{itemize}
    \item[$\infty$] A = Área del cono
    \item[$\infty$] $\pi = 3.1415$
    \item[$\infty$] R = Radio de la base del cono
    \item[$\infty$] g = generatriz
\end{itemize}

$V= \frac{\pi \cdot R^2 \cdot}{3} h$

\begin{itemize}
    \item[$\infty$] V = Volumen del cono
    \item[$\infty$] $\pi = 3.1415$
    \item[$\infty$] R = Radio de la base
    \item[$\infty$] h = Altura
\end{itemize}

$A= \frac{(P+P^1)}{2} \cdot a + A_{B} + A^1_{B}$
\begin{itemize}
    \item[$\Join$] A= Área del tronco de pirámide
    \item[$\Join$] P y $P^1$ = Perímetros de las báses
    \item[$\Join$] a = Apotema
    \item[$\Join$] h = Altura
\end{itemize}

$V= \frac{(A_{B}+A^1_{B}+) \sqrt{A^1 \cdot A^1_{B}}  \cdot h}{3}$

\begin{itemize}
    \item[$\Join$] V = Volumen del tronco de la pirámide
    \item[$\Join$] $A_{B}, A^1_{B}$ = Áreas de las bases 
    \item[$\Join$] h = Altura
\end{itemize}

\subsection{Conversión entre coordenadas cilíndricas y cartesianas}


$x =  r cos \theta$
$y = r sen \theta$
z = z
\begin{itemize}
    \item[$\heartsuit$] Estas ecuaciones se utiizan para convertir de coordenadas cilíndricas a coordenadas rectangulares.
\end{itemize}

$r^2 = x^2+z^2 $
$tan \theta = \frac{y}{x}$
z=z 

\begin{itemize}
    \item[$\heartsuit$] Estas ecuaciones se utiizan para convertir de coordenadas rectangulares a coordenadas cilíndricas.
\end{itemize}








\end{document}
