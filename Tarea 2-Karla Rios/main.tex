\documentclass[12pt]{article}
\usepackage[utf8]{inputenc}
\usepackage{xcolor}
\definecolor{blue}{HTML}{26798E}
\definecolor{green}{HTML}{63CAA7}
\title{Así soy}
\author{Karla Cecilia Valeria Rios Lara }
\date{\today, September 2022}

\begin{document}


\maketitle


\section{Academia}
\subsection{Pasado}
Vengo de la Prepa 8, que se encuentra en Lomas de Plateros en Álvaro Obregón. Cómo yo vivo en Álvaro Obregón me queda relativamente cerca, cómo a 30 min sin tráfico, me quedaba más cerca que la Facultad de Ciencias.



Para ir a la escuela salía de mi casa hacia la parada de los camiones y tomaba un camión que me dajara en la lateral de periferico a la altura de Altavista. De ahí tomaba otro camión que me dejaba justamente a un cruce peatonal de la prepa. De regreso tenía que caminar hasta periférico y de ahí tomar un camión que me dejara en Av. Toluca, de ahi caminar a Tizapan, donde tomaría otro camión que me dejara en mi casa. 



A veces me hacía 40 0 50 min con tráfico de ida y de regreso, tambien dependía de cuanto tardaran en pasar los camiones.

\subsection{Presente}
La Facultad de Ciencias si me queda un poco más lejos de mi casa, a comparación de la prepa pero no demasiado. Ya que Álvaro Obregón y Coyoacan son delegaciones muy cercanas.



Para llegar a la facultad, primero tomo un camión que me deja en Av. Revolución a la altura de la Torre Murano y la Torre Nissan. De ahí camino hasta la Facultad de Psicología y espero el pumabus, la ruta 9, que me deja en el anexo de ingeniería y cruzo por el circuito hacia la facultad.



De regreso tengo que caminar hasta el anexo de ingeniería y tomar el pumabus ruta 9, luego bajarme en psicología y caminar hasta el paradero de San Ángel y tomar el camión que me dejará en mi casa. 



De ida, me hago aproximadamente 50 o 1hr min sin tráfico, para esto igual influye mucho que el camión que me deja en Av. Revolución tarda demasiado en pasar, el tiempo que me haga caminando hacia la parada del pumabus y lo que tarda en pasar el pumabus, porque la ruta 9 tarda siglos en pasar. Con tráfico me he llegado a hacer 1hr y media. De regreso me hago como 40 min, igual dependiendo del pumabus y si hay mucha fila para subir a mi camión.

\subsection{¿Porqué Física?}
Desde pequeña me interesó mucho la ciencia y siempre he sido muy curiosa. Recuerdo que en la primaria mi materia favorita era ciencias naturales, y me gustaba mucho ver documentales de ciencia, especialmente del universo. 



Cuando entré a 2do de secundaria cursé Física, y la materia me gustó mucho, me daba mucha incertidumbre cómo es que todo pasa como pasa, dar una explicación y al mismo tiempo un ¿Por qué? a lo que ocurre a nuestro alrededor con hechos que todos pueden comprobar. 



Siempre me ha encantado aprender cosas nuevas sobre el entendimiento de lo que vivimos a diario, porque la física está en todo y es todo.



Y también una razón del por qué quiero estudiar física es por la seria The big ban theory jaja, es de mis series favoritas.



Finalmente me dí cuenta que la ciencia siempre me ha gustado, y lo que quiero ser es Astrónoma o Astrofísica, aún no decido bien cuál, pero por eso decidí estudiar física para luego hacer mi especialidad en astronomía. 

\section{Hoobies}

\subsection{Natación}
Yo practico natación desde los 15 años y desde que aprendí a nadar, nadar se ha convertido en una de mis grandes pasiones y una de las cosas que más amo hacer en mi vida. Me encanta sentir el agua en mi, me encanta poder sentir cómo floto y todo lo que puedo hacer debajo el agua. Es un deporte tan completo y relajante además. 



Lo practico 2 veces por semana los sábados y domingos.
%un dato curioso es que me da miedo el mar, me encanta nadar pero sólo en alberca, me da miedo perderme en el mar, y también lo que haya debajo de él unu.


\subsection{Reposteria}
Me gusta hacer postres y aprender cómo hacerlos, sé hacer pastel imposible, flan napoilitano, helado, gelatinas de mosaico, crepas, galletas, hotcakes xd, etc..., brownies aún no porque una vez que hice me quedaron cómo galletas y me deprimí jaja. Pero es algo que me gusta mucho y me divierte hacerlo. 



No es algo que le dedique tiempo cada semana, pero cuando lo hago disfruto mucho el proceso y el resultado.

\section{Música preferida}
En realidad me gustá casi de todos los géneros, mis favoritos son los que pondré en la siguiente sección pero también disfruto mucho de géneros cómo electrónica, k-pop, indie, rock en español y sountracks de peliculas y\slash o series, openings y endings de anime
. 
\subsection{Pop ó pop alternativo}
Me gusta escucharlo generalmente cuando me siento tranquila, estoy haciendo mis quehaceres,tareas y también al bañarme, depende de mi estado de ánimo o de mi mood.



\subsubsection{Mis artistas favoritas de estos géneros son:}
Melanie Martínez

-Show \& Tell

-Tag, you´re it

Ariana Gránde 

-7 rings

-No left tears to cry

\subsection{Reggaetón}
Me gusta escucharlo cuando me siento activa y feliz, en las fiestas para bailar, en cualquier momento del día, para bañarme, y cuando me aburro haciendo una tarea. 



\subsubsection{Mis artistas favoritos del género són:}

Feid

-Ron (remix)

-El padrino
%a Feid lo conocí en persona y fué una experiencia increíble, lo pude saludar de mano y escuchar cantar en vivo, fué al primer concierto al que fuí, lo conocí gracias a mi novio ya que él es su fan.

Bad bunny %a lo mejor si es porque está muy de moda pero saca buenas rolas

- Me porto bonito

- Party

\section{Mi experiencia aquí}
La primera semana en la que estuve en la facultad ralmente fue la más dura, tenía expectativas muy diferentes de la universidad, porque pensaba que iba a llegar e iba aprender todo desde 0 y no, tengo compañeros que van mucho más avanzados que yo y me siento pues valga la redundancia demasiado atrasada. Por lo mismo entré en depresión porque me sentía insuficiente para la carrera y pues sentía que había cometido un error, lloré y me sentía muy mal. 



Pero me dí cuenta que no es que no pueda simplemente no sé cosas que los demás saben entonces solamente tengo que aprender, y no darme por vecida, a lo mejor repruebo pero es parte de, y creo que si es algo que quiero y siempre he querido debo luchar y no darme por vencida cuando apenas llevaba una semana. 



Ahora solamente estoy dispuesta a aprender aunque me cueste o me tarde, lo importante es aprender. Amo aprender cosas nuevas todos los días, aunque no entienda al 100 pero lo intento. Tengo muchas ganas, sé que habrá altos y bajos pero así es la vida y nimodo:')

\section{Puntos extras}
\textit

\textbf{-Y todas las noches bajo la via lactea parecen eternas si tu no estas...-}


\textcolor{blue}{-Te besaré hasta sentirme tuyo e inventaré un mundo para los dos}
\textcolor{green}{, te abrazaré y sentirás mi calor haremos sexo con ropa, esto será entre tú y yo bailemos tú y yo...-}

\end{document}
