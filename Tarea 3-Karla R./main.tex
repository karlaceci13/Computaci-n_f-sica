\documentclass[a5paper, 11pt]{article}
\usepackage[utf8]{inputenc}
\usepackage{xcolor}
\usepackage{fancyhdr}
\usepackage{graphicx}
\usepackage{geometry}
\usepackage{caption}
\usepackage{float}
\geometry{top=2.5cm, bottom=3.0cm, left=3.0cm, right=2.5cm}

\pagecolor{black}
\color{white}

\pagestyle{empty}
\pagestyle{plain}
\pagestyle{headings}
\pagestyle{myheadings}
\markright{amarillo}
\pagestyle{fancy}

\lhead{Rios Lara Karla Cecilia Valeria}
\rhead{Un poco de la serie}
\begin{document}


\begin{center}
\textit{\textbf{\huge{The Big Bang Theory}}}    
\end{center}

\section{Un poco de la serie}
\textrm{Es una serie que trata sobre 2 científicos, llamados \emph{Sheldon} y \emph{Leonard}, junto a su vecina Penny, de la cuál está enamorado Leonard. Junto con sus amigos Howard y Raj trabajan en el Caltech, dónde laboran cómo ciéntificos y bastante desctacados, su vida cotidiana es representada cómo una comedia, en la que se muestran las personalidades de los 4 cómo ¨geeks" y con dificultades para llevar acabo su vida social, sobretodo con mujeres, aunque sean unos genios.}

\subsection{Cast}
 \subsubsection{Los mejores}

\begin{itemize}
    \item\textcolor{magenta}{\fcolorbox{white}{blue}{Jim Parsons}}-Sheldon Cooper: Sheldon es un \emph{Físico teórico}, es un genio superdotado desde niño, con IQ de 187, le gusta la ciencia ficción, los comics, videojuegos y los trenes. Es un poco odioso, tiene TOC, no entiende el sarcasmo, no entiende la empatía, lo que lo hace algo insoportable. \emph{Su pareja sentimental es Amy}.
    
    \item \textcolor{magenta}{\fcolorbox{white}{blue} {Johnny Galecki}}-Leonard Hofstadter: Es un \emph{Físico experimental}, es inteligente pero gracias a que sus padres (sobretodo su madre) lo forzaron a ser lo que a ellos les parecía mejor, así buscando aprobación se decidió por la ciencia. Un genio pero al tratarse de Penny se vuelve un tonto. Le gusta la ciencia ficción, los comics, videojuegos y Penny. \emph{Su pareja sentimental es Penny}.
    
    \item \textcolor{magenta}{\fcolorbox{white}{blue} {Kaley Cuoco}}-Penny: Penny es una chica muy linda que viene de Nebraska, que llegó a Pasadena porque su sueño es convertirse en actriz, pero trabaja de messera en ¨The Cheesecake Factory", ella es algo ´´tonta" para los chicos pero los deslumbra con su belleza y carisma, logra entrar en su grupo y convertirse en un miembro muy importante de ellos. \emph{Leonard es su pareja sentimental}.
    
    \item \textcolor{magenta}{\fcolorbox{white}{blue} {Simon Helberg}}-Howard Wolowitz: Howard es un \emph{Ingeniero mecánico}, por lo que Sheldon siempre se burla de que es ingeniero y de no tener un doctorado. El se muestra en la serie cómo el típico espanta mujeres, pervertido y acosador, pero a lo largo de la serie da un gran avance como persona gracias a Bernadette. Le gusta la ciencia ficción, los comics, los videojuegos, las hebillas de cinturón y los cuellos de tortuga. \emph{Bernadette es su pareja sentimental}.
    
    \item \textcolor{magenta}{\fcolorbox{white}{blue}{Kunal Nayyar}}-Raj Koothrappali: Raj es un \emph{astrofísico} extranjero, que viene de una famila bastante adinerada. El no es capáz de hablar con mujeres si no es con alcohol, mucho después supera su trauma. Es tímido, sensible, y muy carismático, nunca ha tenido una buena relación, y las que tuvo las echó a perder. Tiene una relación bastante peculiar con Howard ya que son mejores amigos. Le gusta la ciencia ficción, los comics, los videojuegos, su perrita canela, cocinar y las mujeres.
    
    \item \textcolor{magenta}{\fcolorbox{white}{blue} {Melissa Rauch}}-Bernadette Rostenkowski: Es una \emph{microbióloga} muy linda y temperamental, que conoció a Haward ya que Penny y ella trabajaban juntas en ¨The Cheesecake Factory", Penny los presentó y surgió una realción bastante divertida y que hizo crecer a Haward cómo persona.
    
    \item \textcolor{magenta}{\fcolorbox{white}{blue} {Mayim Bialik}}-Amy Farrah Fowler: Amy es una \emph{neurobióloga}, conoció a Sheldon un día que Haward y Raj le consiguieron una cita por internet a Sheldon, el no quería pero al conocerse en un café, se cayeron muy bien por ser parecidos en no querer tener ni necesitar una relación, ella sólo lo hacía para que su madre no la molestara. Decidieron ser amigos, aunque al pasar del tiempo se dieron cuenta de que ya eran más que eso. Comenzaron una relación un poco rara pero los hizo crecer a ambos. Su mejor amiga es Penny. 
    
\end{itemize}

\subsubsection{Personajes no tan importantes}
\begin{enumerate}
    \item textbf{Kevin Sussman}-Stuart Bloom: Amigo de los chicos y dueño de una \emph{tienda de comics}.
    \item \textbf {Carol Ann Susi}-Sra. Wolowitz: Madre de Haward
    \item textbf{Laurie Metcalf}-Mary Cooper: Madre de Sheldon
    \item \textbf{Christine Baranski}-Beverly Hofstadter: Madre de Leonard
    \item {Wil Wheaton}-Wil Wheaton: Comienza siendo un enemigo para Sheldon por un trauma en su infancia pero luego se convierte en un gran amigo para los chicos.
    \item \textbf{Aarti Mann}-Priya Koothrappali: Hermana de Raj y ex-pareja sentimental de Leonard.
    \item \textbf{Laura Spencer}-Emily Sweeney: Ex-pareja sentimental de Raj.
    \item \textbf {Brian Thomas Smith}-Zac: Ex-pareja sentimental de Penny.
    \item \textbf{John Ross Bowie}-Barry Kripke: Compañero de trabajo de Sheldon, al ser también físico teórico. Le gusta ver a Sheldon sufrir y demostrar que es mejor que él.
    \item \textbf{Brian Posehn}-Bert Kibbler: Compañero de Amy, es un geólogo que comienza pretendiendo a Amy pero cuando se entera de su relación con Sheldon, sólamente es su amigo.
    \item \textbf{Lauren Lapkus}-Denise: Empleada de Stuard, que termina siendo su pareja sentimental.
    \item \textbf{Courtney Henggeler}-Missy Cooper: Hermana de Sheldon.
    \item \textbf{Jerry O'Connell}-George Cooper Jr.:Hermano de Sheldon.
    \item \textbf{Stephen Hawking}- Stephen Hawking: Es invitado especial en los episodios recurrentemente.
\end{enumerate}

\begin{figure}[H]
    \raggedleft
    \caption{La Teoría del Big Bang}
    \includegraphics[scale=0.1, angle=15]{tbbt.jpg}
    \label{fig:my_label}
\end{figure}


\section{Cómo cambió mi vida}

\textcolor{blue}{Para mi es una serie increíble, se me hace muy divertida y aparte aprendes cosas de ciencia quieras o no.}
\textcolor{green}{Te encariñas mucho con los personajes, cómo no es seriado capítulo tras capítulo, es divertido ver cómo es que se van a desarrollar nuevas historias o experiencias para los personajes.} 
\textcolor{purple}{Igual que conforme va avanzando la historia también van evolucionando los personajes, y eso me gusta mucho que no se encasillen en los típicos comportamientos con los que comenzaron. Es una gran serie y muy entretenida, la he visto como unas 6 veces completa, sus 12 temporadas completas y las veces que me faltan por verla de nuevo.}

\section{Personaje favorito}
\begin{figure}[H]
    \raggedleft
    \caption{Stuart}
    \includegraphics[scale=0.5, angle=-18]{bernie.jpg}
    \label{fig:my_label}
\end{figure}

\section{Personaje menos favorito}
\begin{figure}[H]
    \raggedleft
    \caption{Bernadette}
    \includegraphics[scale=0.5, angle=-8]{Stuart.jpg}
    \label{fig:my_label}
\end{figure}


Cuando entiendes
\space{}

\space{las leyes de la física,}

todo es posible













\end{document}
